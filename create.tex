\section{Creating a Matahari Agent}

The best starting point to create a new Matahari agent is to fork the \href{https://github.com/matahari/matahari-agent-example}{example agent} on GitHub. The changes needed to customise the agent for your own purposes are documented in the repository, and this will be kept up to date with any changes in Matahari.

There is already one Matahari Agent outside of the Matahari tree shipping in Fedora 16 that you can also refer to: \href{https://github.com/matahari/libvirt-qmf}{\texttt{libvirt-qmf}}.\footnote{\texttt{libvirt-qmf} is a Matahari agent for \href{http://libvirt.org}{libvirt}. It replaces \texttt{libvirt-qpid}, a QMFv1 Agent that was not integrated with Matahari.}

\newthought{The first requirement} for a new agent is a QMF schema. \href{https://cwiki.apache.org/qpid/qpid-management-framework.html#QpidManagementFramework-Schema}{The schema format is defined} by Apache Qpid, but the Matahari tools generate code for both QMF and D-Bus agents from the same schema.

\begin{figure}[h]
\begin{Verbatim}
<?xml version="1.0" encoding="UTF-8"?>

<schema package="org.matahariproject.testagent">
    <class name="TestClass">
        <property name="hostname" type="sstr" access="RO" desc="Hostname" index="y" />

        <method name="sum" desc="Sum of two numbers">
            <arg name="x" dir="I" type="int32" />
            <arg name="y" dir="I" type="int32" />
            <arg name="result" dir="O" type="int32" />
        </method>
    </class>
</schema>
\end{Verbatim}
\caption{The example agent schema.}
\label{fig:schema}
\end{figure}

A class may comprise properties, statistics and methods.\footnote{Statistics are data that are expected to change rapidly.} As well as the usual integer and string types, properties may include references to other objects.
Methods may have multiple output arguments, and upon failure may throw an Exception that is either a simple string\footnote{The Matahari headers provide a standard list of strings for common errors.} or an arbitrary object with its own schema.

Although the current agents in the Matahari tree all contain only a single object, it is possible both to define multiple classes and to instantiate multiple objects of each class.\footnote{\texttt{libvirt-qmf} gives an example.}

\newthought{QMF agents are} written in C++. With Matahari, most of the required code is either autogenerated from the schema or provided by the MatahariAgent class. You need only to register the schema and to implement the glue code to initialise and update properties and convert QMF method call events into calls to your library.
At present this also includes implementing method dispatch and resolution. The Matahari team hopes to eventually make improvements in this area.

The Matahari agent's main event loop is provided by \texttt{glib}. You can add events such as timers and socket handlers required to interface with your library using \href{http://developer.gnome.org/glib/2.29/glib-The-Main-Event-Loop.html}{functions such as \texttt{g\_timeout\_add()}}.

