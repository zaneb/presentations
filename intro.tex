\section{Introduction to Matahari}

Matahari is a cross-platform\footnote{Linux and Windows are supported so far.} collection of APIs accessible over local and remote interfaces for systems management and monitoring. What the heck does that mean? Read on, dear friends.

\subsection{Architecture Overview}

Matahari APIs are accessible via a few different methods. The core of the functionality that is implemented is done as C libraries. These can be used directly. However, it is expected and intended for most users to access the functionality via one of the agents we provide. A Matahari Agent is an application that provides access to the features implemented in a Matahari Library via some transport. We are currently providing agents for \href{http://www.freedesktop.org/wiki/Software/dbus}{D-Bus} and \href{http://qpid.apache.org/}{QMF}.

D-Bus is used quite heavily as a communications mechanism between applications on a single system. QMF, or the Qpid Management Framework, is used as a remote interface. QMF is a framework for building remote APIs on top of AMQP, an open protocol for messaging.

The agents are generally thin wrappers around a core library, so other transports could be added in the future if the need presents itself. 

\subsection{Current Features}

So, what can you do with Matahari?

Matahari is still under heavy development, but there is already a decent amount of usable functionality. The current set of agents includes:

\begin{itemize}

    \item \emph{Host} - An agent for viewing and controlling hosts.
    \begin{itemize}
        \item View basic host information such as OS, hostname, CPU, RAM, load average, and more.
        \item Host control (shutdown, reboot) 
    \end{itemize}

    \item \emph{Networking} - An agent for viewing and controlling network devices
    \begin{itemize}
        \item Get a list of available network interfaces and information about them, such as IP and MAC addresses
        \item Start and stop network interfaces 
    \end{itemize}

    \item \emph{Services} - An agent for viewing and controlling system services
    \begin{itemize}
        \item List configured services
        \item Start and stop services
        \item Monitor the status of services 
    \end{itemize}

    \item \emph{Sysconfig} - Modify system configuration
    \begin{itemize}
        \item Modify system configuration files (Linux)
        \item Modify system registry (Windows) 
    \end{itemize}

\end{itemize}

\subsection{Use Cases}

An example of a project that already utilizes Matahari is \href{http://pacemaker-cloud.org/}{Pacemaker-cloud}, which is also under heavy development. Pacemaker-cloud utilizes both the Host and Services agents of Matahari. Being able to actively monitor and control services on remote hosts is a key element of being able to provide HA in a cloud deployment.

In addition to providing ready-to-use agents, we also provide some code that makes it easier to write a QMF agent so that third-parties can write their own Matahari agents. One example of this that is already shipping in Fedora 16 is \texttt{libvirt-qmf}, which is a Matahari agent that exposes \href{http://libvirt.org/}{libvirt} functionality over QMF.\footnote{\texttt{libvirt-qmf} replaces \texttt{libvirt-qpid}, a QMFv1 agent that was not integrated with Matahari.}

